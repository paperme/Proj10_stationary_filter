%\documentclass{segabs}
\documentclass[manuscript,ulem,graphix,revised]{geophysics}

\usepackage{amsmath}
\usepackage{graphicx}
%\usepackage{epstopdf}
%\usepackage{slashbox}
\usepackage{hanging}
\usepackage{mathrsfs}
\usepackage{marginnote}
\usepackage{amssymb} 
\usepackage{url}

%\usepackage{tikz}
%\usepackage{geometry}
%\usepackage{setspace}
%\usepackage[margin=0.8in]{geometry}
%\usepackage{float}
%\usepackage{subfig}
%\usepackage{subcaption}

\usepackage{indentfirst}
\makeatletter
\newcommand*{\rom}[1]{\expandafter\@slowromancap\romannumeral #1@}
\makeatother
\begin{document}

\title{A Space-domain stationary filter for P- and S-wave decomposition in isotropic elastic media}

\renewcommand{\thefootnote}{\fnsymbol{footnote}} 

\address{
\footnotemark[1]Department of Mathematics, Harbin Institute of Technology,
92 Xidazhi St., Nangang Dist., Harbin, Heilongjiang, China 150001

\footnotemark[2]Center for Lithospheric Studies, 
The University of Texas at Dallas, \\
800 W Campbell Road (ROC21), Richardson, TX, USA 75080
}

\author{Wenlong Wang\footnotemark[1] and George A. McMechan\footnotemark[2]}

%\footer{Wang \& McMechan}
%\lefthead{Wang \& McMechan}
\righthead{Stationary filter}

\maketitle

%\clearpage
%\newpage
\renewcommand{\figdir}{Fig} % figure directory

\begin{abstract}

P and S wavefields separation/decomposition is essential for generating PP and PS images in elastic reverse time migrations (ERTMs). Traditionally used divergence and curl operators introduce amplitude and phase change of the input elastic wavefield. We propose a novel space domain stationary filter for P- and S-wave vector decomposition. The decomposed P- and S-waves have the same phase and amplitude with the original elastic wavefield. 
We find that the accuracy of decomposition is propotional to the size of the stationary filter. The proposed method is compared with two other PS decomposition algorithms. 
%Elastic reverse time migration (ERTM) is capable of characterizing subsurface properties more completely than its acoustic counterpart. P- and S-waves coexist in elastic wavefields, and their separation is required before, or as part of, applying the image conditions. Traditional P- and S-wave separation methods based on divergence and curl operators don't preserve the elastic vector information, and the associated polarity reversals of S-wave images are difficult to handle. Thus a preferable workflow for isotropic ERTM should include a vector decomposition of the elastic wavefields and a vector-based image condition that directly uses the signed magnitudes of the decomposed vector wavefields to produce PP and PS images. Our vector-based image condition is a source-normalized crosscorrelation image condition, which is robust and stable for generating 3D ERTM images for complicated models. The vector image condition also includes the calculation of propagation directions of the decomposed P- and S-waves, which can be further utilized to efficiently generate PP and PS angle domain common-image gathers.

\end{abstract}

\section{Introduction}
Traditional seismic migrations apply acoustic wave equations, which simulate only P-wave propagation in the earth. However, the existence of S-waves in the records may bring artifacts to the migrated images, 
especially for large offset seismic data, in which the S-waves have comparable amplitutes with the P-waves.
%because the acoustic wave equation cannot handle both wave modes simultaneously. 
Moreover, the S-waves also carry valuable imformation from subsurface, and enable better analysis of the reservoirs. S-waves are also useful for detecting gas clouds and fractures, where the P-waves are strongly attenuated \citep{li98, knapp01}.

%Elastodynamic wave equations can simulate seismic wave propagation with fewer assumptions than acoustic wave equations, as the former include both shear- and compressional wave propagations. 3D elastic migration, with multicomponent data as input, provides the foundation for structure imaging and elastic parameter estimation.

One category for migrating the multicomponent data is the ray-based methods. Examples include Kirchhoff migration \citep{kuo84, dai86, hokstad00}.
%where the P- and S-wave traveltime tables are obtained from ray-tracing, and images are built by summing the amplitudes along their corresponding traveltimes. In this process, the PS separation is implicit by using P- and S-velocity models separately for ray-tracing. 
%The limitations of elastic Kirchhoff migration is the same with those of acoustic Kirchhoff migrations; high frequency assumption makes the ray theory unable to dully describe the wave phenomena especially for complicated models \citep{gray01}. 
Ray-based methods are efficient, but high frequency assumption makes the ray theory unable to dully describe the wave phenomena especially for complicated models \citep{gray01}.
Another category is the wave equations based solutions \citep{chang86,chang94,whitmore95}, which reconstruct the full elastic wavefields from boundary conditions \citep{wapenaar90}. 
%Elastic wave equations directly use multicomponent data as boundary conditions, and reconstruct the elastic wavefield. 
However, P- and S-waves are coupled in the elastic wavefield. Failure to separate the P- and S-waves leads to crosstalk artifacts in the migrated images \citep{yan08}. Thus the quality of PS separation has directly influence on the final images. 

%\citet{sun01} separate multicomponent data near the surface and use acoustic equations for separate PP and PS migrations. \citet{yan08} propose to use elastic wave equations for both source and receiver extrapolations and separate the P- and S-waves by divergence and curl operators before applying the image condition. 
The pervasively applied PS separation method is the divergence and curl operators, which are based on Helmholtz decomposition \citep{aki80} and is highly efficient. However, as pointed out by \citet{sun01}, the phase and amplitude of the separated P- and S-wave are changed by the divergence and curl operators, causing problems for true amplitude imaging and interpretations.
\citet{zhang10} propose to decompose the wavefield into P- and S-wave vectors to preserve their original amplitudes and phases, and call it wavefield decomposition to be differentiated from wavefield separation algorithms which involve divergence and curl and similar operators. 

\citet{zhang10} project wavefields to P- and S-polarization directions. \citet{zhu17} decompose the wavefields by solving a Posson's equation. Both methods need Fourier transform and operate in the wavenumber domain. From another perspective, \citet{ma03} and \citet{wenlong_vct15} introduce an auxiliary stress wavefield into the elastodynamic wave equations \citep{virieux84} to obtain a decoupled P-wavefield. However, artifacts are generated along the sharp velocity interfaces of the model \citep{wenlong_cmp15}. 

% in the wavenumber domain. \citet{cheng14} and \citet{wenlong17} use low-rank approximation to efficiently decompose the P- and S-wave vectors in the wavenumber domain. \citet{zhu17} solve the vector Posson's equations by a fast algorithm in the wavenumber domain to decompose the vector wavefield. 

%PS separation/decomposition algorithms can be classified by their domain of calculations. They can be performed in either space domain or wavenumber domain. 
%The pervasively used divergence and curl operators, which are based on Helmholtz decomposition \citep{aki80}, are space domain filters, and works only in isotropic media. 
%\citet{dellinger90} extend the PS separation algorithm to anisotropic media by projecting the wavefields in the wavenumber domain to P- and S-wave polarization directions, respectively, but this method works only in homogeneous media. \citet{yan09} transform the wavenumber domain operators back to the space domain and propose a non-stationary filter for P and S separation in heterogeneous vertically transversely isotropic (VTI) media. 

%In another prospective, \citet{ma03} and \citet{wenlong_cmp15} introduce an auxiliary P-wave stress wavefield into the elastodynamic wave equations \citep{virieux84} to simulate decoupled P- and S-waves during the wavefield extrapolation, but it generate artifacts if the model containes sharp velocity boundaries \citep{wenlong_cmp15}. 

PS decomposition in the anisotropic media is also possible \citep{zhang10, cheng14, wenlong18}, but we limit our scope to isotropic media. Inspired by \citet{yan09}, we propose a space domain stationary filter to decompose P- and S-waves, which doesn't need Fourier transforms, and the results don't have artifacts along the velocity interfaces.

%It's interesting to see that all the PS vector decomposition algorithms are performed in the wavenumber domain, except for the decoupled wave solution. In this paper, we propose a stationary filter to decompose P- and S-wave vectors directly in the space domain, no need of Fourier transforms. 
The paper is organized as follows: We start by discribing the PS decomposition methodology in the wavenumber domain. Then a space domain approximation of the operators are presented. Next, the proposed algorithm is tested and compared with results from wavenumber domain decomposition \citep{zhang10} and decoupled equations \citep{ma03, wenlong_cmp15}. Finally, several ERTMs are performed with synthetic data using the proposed algorithm for PS decomposition.
%-------------------------------------------------



\section{PS decomposition}
%We follow the procedure introduced by \citet{zhang10}, and first derive the equations for PS vector decomposition in the wavenumber domain. From the Helmholtz decomposition theory, an elasitc vector wavefield $\boldsymbol{U}$ can be decomposed into a curl-free P-wave ($\boldsymbol{U}^P$) and a divergence-free S-wave ($\boldsymbol{U}^S$) \citep{aki80}:
%\begin{equation}
%\boldsymbol{U} = \boldsymbol{U}^P + \boldsymbol{U}^S,
%\label{eqn:helmholtz}
%\end{equation}
%with
%\begin{equation}
%\bigtriangledown \times \boldsymbol{U}^P = \boldsymbol{0},
%\label{eqn:p_space1}
%\end{equation}
%and
%\begin{equation}
%\bigtriangledown \cdot \boldsymbol{U}^S = 0
%\label{eqn:s_space1}
%\end{equation}
%Applying divergence operators on both sides of equation~\ref{eqn:helmholtz} gives
%\begin{equation}
%\bigtriangledown \cdot \boldsymbol{U} =\bigtriangledown \cdot \boldsymbol{U}^P,
%\label{eqn:p_space2}
%\end{equation}
%and applying curl operators on both sides of equation~\ref{eqn:helmholtz} gives
%\begin{equation}
%\bigtriangledown \times \boldsymbol{U} = \bigtriangledown \times \boldsymbol{U}^S.
%\label{eqn:s_space2}
%\end{equation}
%In the wavenumber domain, equations~\ref{eqn:helmholtz}-\ref{eqn:s_space2} become
%\begin{equation}
%\tilde{\boldsymbol{U}} = \tilde{\boldsymbol{U}}^P + \tilde{\boldsymbol{U}}^S,
%\label{eqn:helmholtz2}
%\end{equation}
%\begin{equation}
%\boldsymbol{K} \times \tilde{\boldsymbol{U}}^P = \boldsymbol{0},
%\label{eqn:p_wave1}
%\end{equation}
%\begin{equation}
%\boldsymbol{K} \cdot \tilde{\boldsymbol{U}}^S = 0,
%\label{eqn:s_wave1}
%\end{equation}
%\begin{equation}
%\boldsymbol{K} \cdot \tilde{\boldsymbol{U}} = \boldsymbol{K} \cdot \tilde{\boldsymbol{U}}^P,
%\label{eqn:p_wave2}
%\end{equation}
%and
%\begin{equation}
%\boldsymbol{K} \times \tilde{\boldsymbol{U}} = \boldsymbol{K} \times \tilde{\boldsymbol{U}}^S,
%\label{eqn:s_wave2}
%\end{equation}
%where a tilde over the wavefields indicate wavenumber domain representations, and $\boldsymbol{K}$ is the normalized wavenumber vector, indicating the P-wave polarization direction in isotropic media. 
%
%The decomposed P-wave vectors can be obtained by solving for the linear equations formed by \ref{eqn:p_wave1} and \ref{eqn:p_wave2}. A 2D solution is given by
As indicated by \citet{zhang10}, in the wavenumber domain, the P ($\tilde{\boldsymbol{U}}^P$) and S ($\tilde{\boldsymbol{U}}^S$) wave vectors can be decomposed from the coupled wavefield ($\tilde{\boldsymbol{U}}$) using
\begin{equation}
\tilde{\boldsymbol{U}}^P = \boldsymbol{K} (\boldsymbol{K} \cdot \tilde{\boldsymbol{U}}),
\label{eqn:p_wave}
\end{equation}
and
\begin{equation}
\tilde{\boldsymbol{U}}^S =-\boldsymbol{K} \times (\boldsymbol{K} \times \tilde{\boldsymbol{U}}),
\label{eqn:p_wave}
\end{equation}
respectively, and a tilde over the wavefields indicate wavenumber domain representations. $\boldsymbol{K}$ is the normalized wavenumber vector, indicating the P-wave polarization direction in isotropic media.

In two dimensions, the explicit form of the decomposed P-wave vector is written as
\begin{equation}
\tilde{U}^P_x = K^2_x \tilde{U}_x + K_x K_z \tilde{U}_z,
\label{eqn:p_wave1}
\end{equation}
and 
\begin{equation}
\tilde{U}^P_z = K^2_z \tilde{U}_z + K_z K_x \tilde{U}_x,
\label{eqn:p_wave2}
\end{equation}
where the subscripts $x$ and $z$ represent $x$ and $z$ components of the wavefields. Similarly, the decomposed S-waves have the form
\begin{equation}
\tilde{U}^S_x = K^2_z \tilde{U}_x - K_x K_z \tilde{U}_z,
\label{eqn:s_wave1}
\end{equation}
and 
\begin{equation}
\tilde{U}^S_z = K^2_x \tilde{U}_z + K_z K_x \tilde{U}_x,
\label{eqn:s_wave2}
\end{equation}
or equivalently
\begin{equation}
\tilde{U}^S_x = \tilde{U}_x - \tilde{U}^P_x,
\label{eqn:s_wave3}
\end{equation}
and 
\begin{equation}
\tilde{U}^S_z = \tilde{U}_z - \tilde{U}^P_z.
\label{eqn:s_wave4}
\end{equation}
Thus, only P-wave decomposition is neccessary, as the S-waves can be obtained by a component by component subtraction of the decomposed P-waves from the coupled wavefields in both space and wavenumber domain.

\section{Space domain stationary filters}
\indent\indent
In equations~\ref{eqn:p_wave1} and \ref{eqn:p_wave2}, three wavenumber multipliers $K^2_x$, $K^2_z$ and $K_xK_z$ are involved, which are plotted in Figure~\ref{fig:K_filter.eps}a-c, respectively. 
\plot{K_filter.eps}{width=1.0\columnwidth}
{
Wavenumber domain multiplier $K^2_x$ (a), $K^2_z$ (b) and $K_xK_z$ (c) for isotropic P- and S-wave decomposition.
}
Those wavenumber domain multipliers can be transformed to the space domain as filters. Thus, in the space domain, we can write equivalent expressions to equations~\ref{eqn:p_wave1} and \ref{eqn:p_wave2} as
\begin{equation}
U^P_x = L_x [U_x] + L_{xz} [U_z],
\label{eqn:p_space1}
\end{equation}
and 
\begin{equation}
U^P_z = L_z [U_z] + L_{xz} [U_x],
\label{eqn:p_space2}
\end{equation}
where $L_x$, $L_z$ and $L_{xz}$ are the stationary filters that are designed to decompose the elastic wavefields, and they are calculated by inverse Fourier transforms of $K^2_x$, $K^2_z$ and $K_xK_z$ (Figure~\ref{fig:K_filter.eps}a-c), respectively. The square brackets represent spatial filtering for the wavefield. 

PS decomposition of isotropic wavefields depends only on the polarization directions of P- and S-waves, which have a predetermined relation with wavenumbers, and don't change with model parameters ($V_P$, $V_S$ or $\rho$). 
Thus, the values of filters can be pre-calculated, and only three filters ($L_x$, $L_z$ and $L_{xz}$) need to be stored. The filters can be applied to all grid points in the medium and decompose the wavefields at any time step during wavefield extrapolations. 

A full representation of the wavenumber domain multipliers are infinitely large 2D patches in the space domain. To limit the size of the filters, we follow the strategy of \citet{yan09}, which gives the procedure to transform wavenumber domain operators to the space domain, and evaluates operators at different orders of accuracy. 
%and demonstrated that higher orders give better approximations to the wavenumber domain operators. 
In this paper, we use an uniform eighth order approximations to construct the space domain filters. 

%the  are stationary across all locations of isotropic wavefields. 
A set of pre-calculated $L_x$, $L_z$ and $L_{xz}$ filters are shown in Figures~\ref{fig:filter.eps}a-c. The filters are not as compact, as a result of the wavenumber normalization, which is a deconvolution operator in the space domain. The size of the filters in Figure~\ref{fig:filter.eps} are 51 $\times$ 51, and can be truncated to smaller sizes to improve computational efficiency, but smaller spatial filters lead to lower decomposition accuracy. 
\plot{filter.eps}{width=1.0\columnwidth}
{
Eighth-order $L_x$ (a), $L_z$ (b) and $L_{xz}$ (c) stationary filters. The boxes from inside to outside show the truncation of the size of filters to 9 $\times$ 9, 21 $\times$ 21 and 51 $\times$ 51 for the whole filter (with no truncation).
}

Below, three set of spatial filters of the size 9 $\times$ 9, 21 $\times$ 21 and 51 $\times$ 51 are tested and compared in an elastic wavefield. The model is an isotropic homogeneous model with $V_P$ = 4 km/s, $V_S$ =2.4 km/s, and $\rho$ = 2.5 $\mathrm{g/cm^3}$. The model has 10 m grid spacing in the x- and z-directions. A composite source that generates both P- and S-waves are placed at the center of the model. A Ricker wavelet with a dominant frequency of 10 Hz is used in this example. A snapshot of the x and z particle velocity components at t = 0.34 s is shown in Figure~\ref{fig:homo_snap.eps}
\plot{homo_snap.eps}{width=1.0\columnwidth}
{
(a) The horizontal and (b) vertical component particle velocities in a homogeneous model recorded at time t = 0.34 s.
}

The decomposed P- and S-waves using stationary filters of sizes 9 $\times$ 9, 21 $\times$ 21 and 51 $\times$ 51 are shown in Figure~\ref{fig:homo_size_cmp.eps}. As the S-waves are obtained by subtracting decomposed P-waves from the coupled wavefields, it is easy to see the P-wave residuals in the S-wave pannels, as a result of inadequate filter sizes.
\plot{homo_size_cmp.eps}{width=1.0\columnwidth}
{
The PS decomposition results using stationary filters of sizes 9 $\times$ 9 (a), 21 $\times$ 21 (b) and 51 $\times$ 51 (c), respectively. For each panel, from left to right are the horizontal and vertical components of P-waves and horizontal and vertical components of S-waves.
}
  
We analyze the accuracy of PS decomposition results with different sizes of stationary filters by comparing the residuals between their correspondingly decomposed P-waves and a pure P-wavefield generated with an explosive source at the same time step. The generated P-wave snapshot is shown in Figure~\ref{fig:homo_p_snap.eps}. 
The residuals are plotted as green bars in Figure~\ref{fig:snap_resi_time.eps}, and their corresponding computational time per decomposition are plotted as red bars. The comparison shows that larger filter size gives better decomposition results, but is more computationally expensive.
\plot{homo_p_snap.eps}{width=1.0\columnwidth}
{
(a) The horizontal and (b) vertical component particle velocities of the P-wave generated from a explosive source, and recorded at the same time step as in Figure~\ref{fig:homo_snap.eps}.
}
\plot{snap_resi_time.eps}{width=1.0\columnwidth}
{
The P-wave residuals (green bars) from different filter sizes between the decomposed P-wave and the pure P-waves (Figure~\ref{fig:homo_p_snap.eps}), and their corresponding computational time (red bars) per decomposition.
}

%\section{Operator Properties}
%\indent\indent
%%In this section, we discuss the properties of the spatial stationary filter in terms of order, size, accuracy, and computation efficiency. 
%
%As indicated by \citet{yan09}, the size of the spatial filter has an influence on the accuracy of wavefield decomposition. In this section, three set of spatial filters the size 9 $\times$ 9, 21 $\times$ 21 and 51 $\times$ 51 are tested and compared in a homogeneous model. 
%
\section{Synthetic tests}
\indent\indent
In this section, we perform synthetic tests using the proposed PS decomposition algorithm, and compare the results with other methods

The wavefields are extrapolated with an eighth-order in space, second-order in time, stress-particle-velocity, staggered-grid, finite-difference solution \citep{virieux86}. Convolutional perfectly matched layer (CPML) absorbing boundary conditions \citep{komatitsch07} are used on all four boundaries to reduce unwanted reflections.

Three PS decomposition algorithms are tested: the decoupled propagation (DP), proposed by \citet{ma03}, the wavenumber domain decomposition (WD) by \citet{zhang10}, and the proposed stationary filter decomposition (SF). 

The first test is performed on an isotropic homogeneous model with $V_P$ = 4 km/s, $V_S$ =2.4 km/s, and $\rho$ = 2.5 $\mathrm{g/cm^3}$. The model has 10 m grid spacing in the x- and z-directions. A composite source that generates both P- and S-waves are placed at the center of the model. A Ricker wavelet with a dominant frequency of 10 Hz is used in this example. A snapshot of the x and z particle velocity components at t = 0.34 s is shown in Figure~\ref{fig:homo_snap.eps}
%\plot{homo_snap.eps}{width=1.0\columnwidth}
%{
%(a) The horizontal and (b) vertical component particle velocities in a homogeneous model.
%}
%The decomposed P- and S-waves using DP, WD and FS methods are shown in Figure~\ref{fig:homo_cmp.eps}
%\plot{homo_cmp.eps}{width=1.0\columnwidth}
%{
%The PS decomposition results using DP (a), WD (b) and FS (c).
%}
As in decomposition algorithms, the S-waves can be obtained by subtracting P-waves from the coupled wavefields component by component, so ...


\section{Discussion}
\indent\indent
The idea of using spatial filers for PS decomposition can be extended to anisottopic media. Unlike isotropic wavefields, in anisotrpic wavefields, the polarization directions of P- and S-waves depends on the model parameters, which makes the spatial filters unstationary across different positions within the model. 



\section{Conclusions}



\section{Acknowledgments}

The research leading to this paper is supported by the Sponsors of the
UT-Dallas Geophysical Consortium and the Outstanding Young Talent Program (AUGA5710053217) from the Harbin Institute of Technology. A portion of the computations were done at the Texas Advanced Computing Center. This paper is Contribution No. xxxx
from the Department of Geosciences at the University of Texas at Dallas.



\newpage

\bibliographystyle{seg}  % style file is seg.bst
\bibliography{att}

\end{document}
