%\documentclass{segabs}
\documentclass[manuscript,ulem,graphix,revised]{geophysics}

\usepackage{amsmath}
\usepackage{graphicx}
%\usepackage{epstopdf}
%\usepackage{slashbox}
\usepackage{hanging}
\usepackage{mathrsfs}
\usepackage{marginnote}
\usepackage{amssymb} 
\usepackage{url}

%\usepackage{tikz}
%\usepackage{geometry}
%\usepackage{setspace}
%\usepackage[margin=0.8in]{geometry}
%\usepackage{float}
%\usepackage{subfig}
%\usepackage{subcaption}

\usepackage{indentfirst}
\makeatletter
\newcommand*{\rom}[1]{\expandafter\@slowromancap\romannumeral #1@}
\makeatother
\begin{document}

\title{A Space-domain stationary filter for P- and S-wave decomposition in isotropic elastic media}

\renewcommand{\thefootnote}{\fnsymbol{footnote}} 

\address{
\footnotemark[1]Department of Mathematics, Harbin Institute of Technology,
92 Xidazhi St., Nangang Dist., Harbin, Heilongjiang, China 150001

\footnotemark[2]Center for Lithospheric Studies, 
The University of Texas at Dallas, \\
800 W Campbell Road (ROC21), Richardson, TX, USA 75080
}

\author{Wenlong Wang\footnotemark[1] and George A. McMechan\footnotemark[2]}

%\footer{Wang \& McMechan}
%\lefthead{Wang \& McMechan}
\righthead{Stationary filter}

\maketitle

%\clearpage
%\newpage
\renewcommand{\figdir}{Fig} % figure directory

\begin{abstract}

P and S wavefields separation/decomposition is essential for generating PP and PS images in elastic reverse time migrations (ERTMs). Traditionally used divergence and curl operators introduce amplitude and phase change of the input elastic wavefield. We propose a novel space domain stationary filter for P- and S-wave vector decomposition. The decomposed P- and S-waves have the same phase and amplitude with the original elastic wavefield. 
We find that the accuracy of decomposition is propotional to the size of the stationary filter. The proposed method is compared with two other PS decomposition algorithms. 
%Elastic reverse time migration (ERTM) is capable of characterizing subsurface properties more completely than its acoustic counterpart. P- and S-waves coexist in elastic wavefields, and their separation is required before, or as part of, applying the image conditions. Traditional P- and S-wave separation methods based on divergence and curl operators don't preserve the elastic vector information, and the associated polarity reversals of S-wave images are difficult to handle. Thus a preferable workflow for isotropic ERTM should include a vector decomposition of the elastic wavefields and a vector-based image condition that directly uses the signed magnitudes of the decomposed vector wavefields to produce PP and PS images. Our vector-based image condition is a source-normalized crosscorrelation image condition, which is robust and stable for generating 3D ERTM images for complicated models. The vector image condition also includes the calculation of propagation directions of the decomposed P- and S-waves, which can be further utilized to efficiently generate PP and PS angle domain common-image gathers.

\end{abstract}

\section{Introduction}
Traditional seismic migrations apply acoustic wave equations, which simulate only P-wave propagation in the earth. However, the existence of S-waves in the records may bring artifacts to the migrated images, 
especially for large offset seismic data, in which the S-waves have comparable amplitutes with the P-waves.
%because the acoustic wave equation cannot handle both wave modes simultaneously. 
Moreover, the S-waves also carry valuable imformation from subsurface, and enable better analysis of the reservoirs. S-waves are also useful for detecting gas clouds and fractures, where the P-waves are strongly attenuated \citep{li98, knapp01}.

%Elastodynamic wave equations can simulate seismic wave propagation with fewer assumptions than acoustic wave equations, as the former include both shear- and compressional wave propagations. 3D elastic migration, with multicomponent data as input, provides the foundation for structure imaging and elastic parameter estimation.

One category for migrating the multicomponent data is the ray-based methods. Examples include Kirchhoff migration \citep{kuo84, dai86, hokstad00}.
%where the P- and S-wave traveltime tables are obtained from ray-tracing, and images are built by summing the amplitudes along their corresponding traveltimes. In this process, the PS separation is implicit by using P- and S-velocity models separately for ray-tracing. 
%The limitations of elastic Kirchhoff migration is the same with those of acoustic Kirchhoff migrations; high frequency assumption makes the ray theory unable to dully describe the wave phenomena especially for complicated models \citep{gray01}. 
Ray-based methods are efficient, but high frequency assumption makes the ray theory unable to dully describe the wave phenomena especially for complicated models \citep{gray01}.
Another category is the wave equations based solutions \citep{chang86,chang94,whitmore95}, which reconstruct the full elastic wavefields from boundary conditions \citep{wapenaar90}. 
%Elastic wave equations directly use multicomponent data as boundary conditions, and reconstruct the elastic wavefield. 
However, P- and S-waves are coupled in the elastic wavefield. Failure to separate the P- and S-waves leads to crosstalk artifacts in the migrated images \citep{yan08}. Thus the quality of PS separation has directly influence on the final images. 

%\citet{sun01} separate multicomponent data near the surface and use acoustic equations for separate PP and PS migrations. \citet{yan08} propose to use elastic wave equations for both source and receiver extrapolations and separate the P- and S-waves by divergence and curl operators before applying the image condition. 
The pervasively applied PS separation method is the divergence and curl operators, which are based on Helmholtz decomposition \citep{aki80} and is highly efficient. However, as pointed out by \citet{sun01}, the phase and amplitude of the separated P- and S-wave are changed by the divergence and curl operators, causing problems for true amplitude imaging and interpretations.
\citet{zhang10} propose to decompose the wavefield into P- and S-wave vectors to preserve their original amplitudes and phases, and call it wavefield decomposition to be differentiated from wavefield separation algorithms which involve divergence and curl and similar operators. 

\citet{zhang10} project wavefields to P- and S-polarization directions. \citet{zhu17} decompose the wavefields by solving a Posson's equation. Both methods need Fourier transform and operate in the wavenumber domain. From another perspective, \citet{ma03} and \citet{wenlong_vct15} introduce an auxiliary stress wavefield into the elastodynamic wave equations \citep{virieux84} to obtain a decoupled P-wavefield. However, artifacts are generated along the sharp velocity interfaces of the model \citep{wenlong_cmp15}. 

% in the wavenumber domain. \citet{cheng14} and \citet{wenlong17} use low-rank approximation to efficiently decompose the P- and S-wave vectors in the wavenumber domain. \citet{zhu17} solve the vector Posson's equations by a fast algorithm in the wavenumber domain to decompose the vector wavefield. 

%PS separation/decomposition algorithms can be classified by their domain of calculations. They can be performed in either space domain or wavenumber domain. 
%The pervasively used divergence and curl operators, which are based on Helmholtz decomposition \citep{aki80}, are space domain filters, and works only in isotropic media. 
%\citet{dellinger90} extend the PS separation algorithm to anisotropic media by projecting the wavefields in the wavenumber domain to P- and S-wave polarization directions, respectively, but this method works only in homogeneous media. \citet{yan09} transform the wavenumber domain operators back to the space domain and propose a non-stationary filter for P and S separation in heterogeneous vertically transversely isotropic (VTI) media. 

%In another prospective, \citet{ma03} and \citet{wenlong_cmp15} introduce an auxiliary P-wave stress wavefield into the elastodynamic wave equations \citep{virieux84} to simulate decoupled P- and S-waves during the wavefield extrapolation, but it generate artifacts if the model containes sharp velocity boundaries \citep{wenlong_cmp15}. 

PS decomposition in the anisotropic media is also possible \citep{zhang10, cheng14, wenlong18}, but we limit our scope to isotropic media. Inspired by \citet{yan09}, we propose a space domain stationary filter to decompose P- and S-waves, which doesn't need Fourier transforms, and the results don't have artifacts along the velocity interfaces.

%It's interesting to see that all the PS vector decomposition algorithms are performed in the wavenumber domain, except for the decoupled wave solution. In this paper, we propose a stationary filter to decompose P- and S-wave vectors directly in the space domain, no need of Fourier transforms. 
The paper is organized as follows: We start by discribing the PS decomposition methodology in the wavenumber domain. Then a space domain approximation of the operators are presented. Next, the proposed algorithm is tested and compared with results from wavenumber domain decomposition \citep{zhang10} and decoupled equations \citep{ma03, wenlong_cmp15}. Finally, several ERTMs are performed with synthetic data using the proposed algorithm for PS decomposition.
%-------------------------------------------------



%Different image conditions are applied in ERTMs. Early attempts include the excitation time image condition \citep{chang86}, in which the image time is calculated by raytracing from the source point. The crosscorrelation image condition \citep{claerbout85} remains the standard image condition for acoustic RTMs, but in ERTMs, a component-by-component crosscorrelation causes crosstalk between the unseparated P- and S-waves leads to artifacts which pose difficulties in interpretation. \citet{yan08} apply divergence and curl operators to separate the P- and S-waves before applying the crosscorrelation image condition and demonstrate improvements in image quality. The PS image with curl operators have a polarity reversal problem and needs special treatment \citep{du12} to avoid destructive interference in post-migration stacking.

%Recently, P- and S-wave vector decomposition \citep{ma03,zhang07,wenlong_cmp15,wenlong_pv16,zhu17,wenlong17} is gaining popularity. Vector decomposition preserves the vector information in the input elastic wavefield and retains the same physical magnitude and units, and thus is considered to be more accurate than using divergence and curl operators which change the phase and amplitude of the input wavefield. \citet{wang_cl16} obtain decomposed elastic vectors from a source and receiver wavefield and use a dot product type of crosscorrelation image condition to generate images. The dot product, however, leads to image amplitude changes as a function of the open angle between the incident and reflected (converted) waves, causing difficulties in AVA analysis. \citet{wenlong_vct15,wenlong_3d16} use the signed magnitude ratio to construct images by a 2D (or 3D) excitation amplitude image condition. 
%This image condition applies Poynting vectors \citep{cerveny01} calculated from decomposed P- and S-waves to obtain angle-domain common-image gathers (ADCIGs). The excitation amplitude image condition features a significant reduction in the source wavefield storage and the I/O burden, but it requires sophisticated improvements to include multipathing \citep{jin15}, which is common in complicated models. In this paper, we extend the vector-based image condition to a more robust source-normalized crosscorrelation type image condition and apply it to 3D ERTMs.

%P- and S-wave vector decomposition is also possible in anisotropic wavefields \citep{cheng14,wenlong17}. However, we limit the scope of this initial elastic paper to isotropic media, as a main goal is the proof of concept. The paper is organized as follows; first the methodology for obtaining decomposed P- and S-wave vectors is described. Then we illustrate the procedure for implementing the 3D vector-based source-normalized crosscorrelation image condition.
%the relation between our image condition and dot product image condition is discussed. 
%The proposed 3D ERTM procedure is successfully tested on a single-layer model and a portion of the SEG/EAGE Overthrust model \citep{aminzadeh94}. 
\section{Methodology}
%We follow the procedure introduced by \citet{zhang10}, and first derive the equations for PS vector decomposition in the wavenumber domain. From the Helmholtz decomposition theory, an elasitc vector wavefield $\boldsymbol{U}$ can be decomposed into a curl-free P-wave ($\boldsymbol{U}^P$) and a divergence-free S-wave ($\boldsymbol{U}^S$) \citep{aki80}:
%\begin{equation}
%\boldsymbol{U} = \boldsymbol{U}^P + \boldsymbol{U}^S,
%\label{eqn:helmholtz}
%\end{equation}
%with
%\begin{equation}
%\bigtriangledown \times \boldsymbol{U}^P = \boldsymbol{0},
%\label{eqn:p_space1}
%\end{equation}
%and
%\begin{equation}
%\bigtriangledown \cdot \boldsymbol{U}^S = 0
%\label{eqn:s_space1}
%\end{equation}
%Applying divergence operators on both sides of equation~\ref{eqn:helmholtz} gives
%\begin{equation}
%\bigtriangledown \cdot \boldsymbol{U} =\bigtriangledown \cdot \boldsymbol{U}^P,
%\label{eqn:p_space2}
%\end{equation}
%and applying curl operators on both sides of equation~\ref{eqn:helmholtz} gives
%\begin{equation}
%\bigtriangledown \times \boldsymbol{U} = \bigtriangledown \times \boldsymbol{U}^S.
%\label{eqn:s_space2}
%\end{equation}
%In the wavenumber domain, equations~\ref{eqn:helmholtz}-\ref{eqn:s_space2} become
%\begin{equation}
%\tilde{\boldsymbol{U}} = \tilde{\boldsymbol{U}}^P + \tilde{\boldsymbol{U}}^S,
%\label{eqn:helmholtz2}
%\end{equation}
%\begin{equation}
%\boldsymbol{K} \times \tilde{\boldsymbol{U}}^P = \boldsymbol{0},
%\label{eqn:p_wave1}
%\end{equation}
%\begin{equation}
%\boldsymbol{K} \cdot \tilde{\boldsymbol{U}}^S = 0,
%\label{eqn:s_wave1}
%\end{equation}
%\begin{equation}
%\boldsymbol{K} \cdot \tilde{\boldsymbol{U}} = \boldsymbol{K} \cdot \tilde{\boldsymbol{U}}^P,
%\label{eqn:p_wave2}
%\end{equation}
%and
%\begin{equation}
%\boldsymbol{K} \times \tilde{\boldsymbol{U}} = \boldsymbol{K} \times \tilde{\boldsymbol{U}}^S,
%\label{eqn:s_wave2}
%\end{equation}
%where a tilde over the wavefields indicate wavenumber domain representations, and $\boldsymbol{K}$ is the normalized wavenumber vector, indicating the P-wave polarization direction in isotropic media. 
%
%The decomposed P-wave vectors can be obtained by solving for the linear equations formed by \ref{eqn:p_wave1} and \ref{eqn:p_wave2}. A 2D solution is given by
As indicated by \citet{zhang10}, the P ($\tilde{\boldsymbol{U}}^P$) and S ($\tilde{\boldsymbol{U}}^S$) wave vectors can be decomposed from the coupled wavefield ($\tilde{\boldsymbol{U}}$) using
\begin{equation}
\tilde{\boldsymbol{U}}^P = \boldsymbol{K} (\boldsymbol{K} \cdot \tilde{\boldsymbol{U}}),
\label{eqn:p_wave}
\end{equation}
and
\begin{equation}
\tilde{\boldsymbol{U}}^S =-\boldsymbol{K} \times (\boldsymbol{K} \times \tilde{\boldsymbol{U}}),
\label{eqn:p_wave}
\end{equation}
respectively, and a tilde over the wavefields indicate wavenumber domain representations. $\boldsymbol{K}$ is the normalized wavenumber vector, indicating the P-wave polarization direction in isotropic media.

In two dimensions, the explicit form of the decomposed P-wave vector is written as
\begin{equation}
\tilde{U}^P_x = K^2_x \tilde{U}_x + K_x K_z \tilde{U}_z,
\label{eqn:p_wave1}
\end{equation}
and 
\begin{equation}
\tilde{U}^P_z = K^2_z \tilde{U}_z + K_z K_x \tilde{U}_x,
\label{eqn:p_wave2}
\end{equation}
where the subscripts $x$ and $z$ represent $x$ and $z$ components of the wavefields. Similarly, the decomposed S-waves have the form
\begin{equation}
\tilde{U}^S_x = K^2_z \tilde{U}_x - K_x K_z \tilde{U}_z,
\label{eqn:s_wave1}
\end{equation}
and 
\begin{equation}
\tilde{U}^S_z = K^2_x \tilde{U}_z + K_z K_x \tilde{U}_x,
\label{eqn:s_wave2}
\end{equation}
or equivalently
\begin{equation}
\tilde{U}^S_x = \tilde{U}_x - \tilde{U}^P_x,
\label{eqn:s_wave3}
\end{equation}
and 
\begin{equation}
\tilde{U}^S_z = \tilde{U}_z - \tilde{U}^P_z.
\label{eqn:s_wave4}
\end{equation}
Thus, only P-wave decomposition is neccessary, as the S-waves can be obtained by a component by component subtraction of the decomposed P-waves from the coupled wavefields in both space and wavenumber domain.

In the space domain, we can write an equivalent expression to equations~\ref{eqn:p_wave1} and \ref{eqn:p_wave2}
\begin{equation}
U^P_x = L_x [U_x] + L_{xz} [U_z],
\label{eqn:p_space1}
\end{equation}
and 
\begin{equation}
U^P_z = L_z [U_z] + L_{xz} [U_x],
\label{eqn:p_space2}
\end{equation}
where $L_x$, $L_z$ and $L_{xz}$ are inverse Fourier transforms of $K^2_x$, $K^2_z$ and $K_xK_z$, respectively. The square brackets represent spatial filtering for the wavefield. The values of $L_x$, $L_z$ and $L_{xz}$ can  be pre-calculated and are stationary across all locations of isotropic wavefields. A 21 by 21 grid example of the $L_x$, $L_z$ and $L_{xz}$ filters are shown in Figure~\ref{fig:filter.eps}.

\plot{filter.eps}{width=1.0\columnwidth}
{
A 21 by 21 grid example of the $L_x$ (a), $L_z$ (b) and $L_{xz}$ (c) filters, respectively.
}



%The procedure of 3D vector-based source-normalized crosscorrelation ERTM is similar to that of the 2D vector-based excitation amplitude image condition \citep{wenlong_vct15}. In ERTMs, the source wavefield is extrapolated before the receiver wavefield extrapolation. Both elastic wavefield extrapolations involve P- and S-wavefield decompositions. The decomposed P- and S-wave particle-velocities and stresses are used to obtain their propagation directions and reflection polarities. 
%Due to the different polarizations of PP and PS reflections, their image conditions are also different. 
%In the following subsections, the procedures are explained and illustrated in detail.

%\subsection{3D elastic wavefield extrapolation and PS decomposition}
%
%We follow the stress-particle-velocity formulation proposed by Madariaga (\citeyear{madariaga76}) and Virieux (\citeyear{virieux84}, \citeyear{virieux86}), which includes the general Hooke's law
%\begin{subequations}
%\begin{equation}
%\frac{\partial\tau_{xx}}{\partial t}=(\lambda+2\mu)(\frac{\partial v_x}{\partial x}+\frac{\partial v_y}{\partial y}+\frac{\partial v_z}{\partial z})-2\mu(\frac{\partial v_y}{\partial y}+\frac{\partial v_z}{\partial z}),
%\end{equation}
%\begin{equation}
%\frac{\partial\tau_{yy}}{\partial t}=(\lambda+2\mu)(\frac{\partial v_x}{\partial x}+\frac{\partial v_y}{\partial y}+\frac{\partial v_z}{\partial z})-2\mu(\frac{\partial v_x}{\partial x}+\frac{\partial v_z}{\partial z}),
%\end{equation}
%\begin{equation}
%\frac{\partial\tau_{zz}}{\partial t}=(\lambda+2\mu)(\frac{\partial v_x}{\partial x}+\frac{\partial v_y}{\partial y}+\frac{\partial v_z}{\partial z})-2\mu(\frac{\partial v_x}{\partial x}+\frac{\partial v_y}{\partial y}),
%\end{equation}
%\begin{equation}
%\frac{\partial\tau_{xy}}{\partial t}=\mu(\frac{\partial v_x}{\partial y}+\frac{\partial v_y}{\partial x}),
%\end{equation}
%\begin{equation}
%\frac{\partial\tau_{xz}}{\partial t}=\mu(\frac{\partial v_x}{\partial z}+\frac{\partial v_z}{\partial x}),
%\end{equation}
%\begin{equation}
%\frac{\partial\tau_{yz}}{\partial t}=\mu(\frac{\partial v_y}{\partial z}+\frac{\partial v_z}{\partial y}), 
%\end{equation}
%\label{eqn:stress-velocity1}
%\end{subequations}
%and the equations of motion
%\begin{subequations}
%\begin{equation}
%\rho\frac{\partial v_x}{\partial t}=\frac{\partial\tau _{xx}}{\partial x}+\frac{\partial\tau _{xy}}{\partial y}+\frac{\partial\tau _{xz}}{\partial z},
%\end{equation}
%\begin{equation}
%\rho\frac{\partial v_y}{\partial t}=\frac{\partial\tau _{xy}}{\partial x}+\frac{\partial\tau _{yy}}{\partial y}+\frac{\partial\tau _{yz}}{\partial z},
%\end{equation}
%%\text{and}
%\begin{equation}
%\rho\frac{\partial v_z}{\partial t}=\frac{\partial\tau _{xz}}{\partial x}+\frac{\partial\tau _{yz}}{\partial y}+\frac{\partial\tau _{zz}}{\partial z},
%\end{equation}
%\label{eqn:stress-velocity2}
%\end{subequations}
%\noindent{where $\tau$ is stress and $v$ is particle-velocity, and the subscripts indicate $x$, $y$ or $z$ coordinates.}
%
%%PS decomposition is necessary to get clear PP and PS images free from artifacts. 
%The above stress-particle-velocity formulation extrapolates 3D isotropic elastic wavefields with coupled P- and S-waves; if they are not separated either before extrapolation, or as part of the image condition, they will be superimposed as 'crosstalk' artifacts in the migrated images. Instead of using curl and divergence, which produce scalar and vector potentials rather than particle vector components \citep{aki80,yan08}, we decompose the wavefields while preserving their vector information. This can be achieved by calculating an auxiliary P-wave stress $\tau^p$, which is a scalar wavefield similar to pressure in the acoustic wave equation, while solving the complete stress-particle-velocity formulation in equations \ref{eqn:stress-velocity1} and \ref{eqn:stress-velocity2}. The calculation for P-wave stress has a scalar form \citep{xiao10} 
%\begin{equation}
%\frac{\partial\tau_{p}}{\partial t}=(\lambda+2\mu)(\frac{\partial v_x}{\partial x}+\frac{\partial v_y}{\partial y}+\frac{\partial v_z}{\partial z}).
%\label{eqn:tau-p}
%\end{equation}
%Then the horizontal and vertical particle-velocity components of P-waves, $v^p_{x}$, $v^p_{y}$ and $v^p_{z}$, are calculated from $\tau^p$ by finite differencing
%\begin{subequations}
%\begin{equation}
%\frac{\partial v^p_{x}}{\partial t}=\frac{1}{\rho}\frac{\partial\tau _{p}}{\partial x},
%\end{equation}
%\begin{equation}
%\frac{\partial v^p_{y}}{\partial t}=\frac{1}{\rho}\frac{\partial\tau _{p}}{\partial y},
%\end{equation}
%\text{and}
%\begin{equation}
%\frac{\partial v^p_{z}}{\partial t}=\frac{1}{\rho}\frac{\partial\tau _{p}}{\partial z}.
%\end{equation}
%\label{eqn:vp}
%\end{subequations}
%This gives a complete description of the vector P-wavefield; the S-wavefield can be obtained by subtracting the P-wavefield from the complete wavefield, component-by-component
%\begin{subequations}
%\begin{equation}
%v^s_{x}=v_x-v^p_{x},
%\end{equation}
%\begin{equation}
%v^s_{y}=v_y-v^p_{y},
%\end{equation}
%\text{and}
%\begin{equation}
%v^s_{z}=v_z-v^p_{z},
%\end{equation}
%\label{eqn:vs}
%\end{subequations}
%where $v^s_x$,  $v^s_y$ and $v^s_z$ are the $x$, $y$ and $z$ direction particle-velocity components of the S-waves. 
%
%An example of elastic wavefield decomposition is shown in Figure~\ref{fig:homo_snap.eps}, where a composite (P and S) source is placed at the center of a homogeneous model with propagation velocities vp = 2.5 km/s, vs= 1.6 km/s, and $\rho$ = 2.1 $\mathrm{kg/cm^3}$. The first row are the x, y and z components of the original elastic wavefield snapshot (Figure~\ref{fig:homo_snap.eps}a-\ref{fig:homo_snap.eps}c); the second and the third rows are the corresponding decomposed components of the P (Figure~\ref{fig:homo_snap.eps}d-\ref{fig:homo_snap.eps}f) and the S (Figure~\ref{fig:homo_snap.eps}g-\ref{fig:homo_snap.eps}i) waves with vector components preserved. 
%\plot{homo_snap.eps}{width=1.0\columnwidth}
%{
%An elastic wavefield decomposition example in a homogeneous model.
%(a)-(c) are the X, Y and Z components of the original particle-velocity components;
%(d)-(f) are the X, Y and Z components of the decomposed P-wave particle-velocity components;
%(g)-(i) are the X, Y and Z components of the decomposed S-wave particle-velocity components;
%}




\section{Synthetic Tests}


\section{Discussion}

The proposed vector-based source-normalized crosscorrelation image condition has several important differences from the source-normalized crosscorrelation image condition in acoustic RTMs \citep{kaelin06}. First, the magnitudes of the elastic particle-velocity vectors are used instead of amplitudes in acoustic wavefields. Second, for acoustic RTMs, only a single component amplitude needs to be stored during the source wavefield extrapolation, while in vector-based image condition for elastic RTM, all the P-wave particle-velocity and stress components are required at each grid point to build the final image. Third, because the magnitudes of the particle-velocity vectors are always positive, to get accurate reflectivity information, signs of the reflections (both PP and PS) need to be determined as a part of the image condition. The output of the vector-based image condition contains approximate angle-dependent reflectivity information; to increase the accuracy of the reflectivity, compensations for transmission and attenuation losses \citep{deng07,deng08} are also necessary.

In the above examples, we store the source wavefield in disk as the models are small. However, for big 3D models, the wavefield storage and I/O burden may become costly. One solution is to store the boundaries or checkpoints during the source wavefield extrapolation \citep{bao15}, and reconstruct the source wavefield during the receiver wavefield extrapolation.

Compared with the excitation type image condition \citep{wenlong_vct15,wenlong_3d16}, the crosscorrelation type image condition involves substantially more wavefield storage and a large I/O burden, and thus is more computationally expensive; the main benefit is that multipathing can be handled efficiently and correctly.

The proposed image condition uses the relation between P- and S-wave directions and their corresponding polarization directions, which is valid in isotropic media. In anisotropic media, the relation is more complicated, and an extension of this image condition to anisotropic ERTM needs further investigations.


\section{Conclusions}

A vector-based source-normalized crosscorrelation image condition is proposed and implemented in 3D ERTMs. P- and S-waves are decomposed in the vector domain during both source and receiver wavefield extrapolation. Propagation directions for P- and S-waves are efficiently calculated using Poynting vectors with the decomposed P- and S-wave vectors as input, and make the process of generating ADCIGs much cheaper compared with other existing methods that involve extracting propagation directions from wavefronts. 


\section{Acknowledgments}

The research leading to this paper is supported by the Sponsors of the
UT-Dallas Geophysical Consortium and the Outstanding Young Talent Program (AUGA5710053217) from the Harbin Institute of Technology. A portion of the computations were done at the Texas Advanced Computing Center. This paper is Contribution No. xxxx
from the Department of Geosciences at the University of Texas at Dallas.



\newpage

\bibliographystyle{seg}  % style file is seg.bst
\bibliography{att}

\end{document}
