\documentclass[aspectratio=43]{beamer}
\usetheme{Boadilla}
%\usepackage{slashbox}
%\usepackage{hanging}
%\usepackage{algorithm,algorithmic}
\usepackage{amsmath}
\usepackage{tikz}
\usepackage{fancybox}
\usepackage{mathrsfs}
\usepackage{makecell}
\usepackage{listings}
%\usepackage[rightcaption]{sidecap}
%\usepackage{enumitem}
% set first line indentation
\parindent=20pt
\setbeamersize{text margin left=20pt,text margin right=20pt}

\logo {\includegraphics[height=0.8cm]{./Fig/hitlogo.eps}}
% items enclosed in square brackets are optional; explanation below
\title[PS decomposition]{Space-domain stationary filters for PS decomposition in isotropic media}
%\subtitle[subs]{Parallelization with domain decomposition}
\author[W. Wang]{Wenlong Wang \inst{1}}
\titlegraphic{\includegraphics[height=0.8cm]{./Fig/hitlogo.eps}}

\institute[HIT]{\inst{1} Harbin Institute of Technology}
\date[Dec 2017]{Dec, 2017}

\begin{document}

% remove figure caption
\setbeamertemplate{caption}{\insertcaption}


%--- the titlepage frame -------------------------%
\begin{frame}[plain]
  \titlepage
\end{frame}

\begin{frame}
\frametitle{Outline}
\tableofcontents
%\tableofcontents[part=1,pausesections]
\end{frame}
%-----------------------------------------------------------
\section{Motivations}
\begin{frame}
\frametitle{Outline}
\tableofcontents[currentsection]
\end{frame}
\begin{frame}
  \frametitle{Motivations}
\begin{itemize}
\item{P and S wavefields separation/decomposition is essential for generating PP and PS images in elastic RTMs.}
\item{Classical divergence and curl operators cause phase and amplitude change after PS separation.}
\item{Wavenumber domain methods are difficult to implement with parallel computation because of large scale Fourier transforms.}
\item{Decoupled equation method causes artifacts at reflection interfaces.}
\end{itemize}
\end{frame}
%-----------------------------------------------------------
\section{Previous work}
\begin{frame}
  \frametitle{Previous work}
\tiny{
\begin{itemize}
\item{Dellinger and Etgen (1990) propose divergence-like and curl-like operators in the wavenumber domain based on Christoffel equation to perform separation in anisotropic homogeneous elastic media.}
\item{Ma and Zhu (2003) solve the decoupled elastic wave equations for decomposed P and S waves with pseudospectrum formulation.}
\item{Yan and Sava (2008) transform the wavenumber domain separation operator from the wavenumber domain to the space domain, and formulate nonstationary filter to separate HTI wavefields.}
\item{Zhang and McMechan (2010) realize the PS decomposition by generalizing Dellinger and Etgen's (1990) theory and applied to VTI wavefields.}
\item{Cheng and Fomel (2014) use a low-rank approximation to build space-wavenumber mixed-domain operators and significantly reduce the number of FFT operations per time step.}
\item{Wang et al. (2015) use decoupled equation to build PP and PS images with vector-based image condition.} 
\item{Wang et al. (2017) propose localized a localized low-rank approximation to improve efficiency and accuracy for PS decompositions in anisotropic media.}  
\end{itemize}
}
\end{frame}
%-----------------------------------------------------------
\begin{frame}{The idea}
We formulate stationary filters to decompose isotropic P and S waves in the space domain, which don't require Fourier transforms or decoupled elastic equations.
\end{frame}
%%-----------------------------------------------------------
\section{Methodology}
\begin{frame}{Methodology}

\begin{columns}
  \begin{column}{0.50\textwidth}

\begin{block}{Separation (space)}
\begin{equation}
\begin{aligned}
P&=\bigtriangledown \cdot \boldsymbol{U} \\
\boldsymbol{S}&=\bigtriangledown \times \boldsymbol{U}
\end{aligned}
\end{equation}
\end{block}

\begin{block}{Separation (wavenumber)}
\begin{equation}
\begin{aligned}
\tilde{P}&=i\mathrm{\boldsymbol{k}} \cdot \boldsymbol{\tilde{U}} \\
\boldsymbol{\tilde{S}}&=i \mathrm{\boldsymbol{k}} \times \boldsymbol{\tilde{U}}
\end{aligned}
\end{equation}
\end{block}
\end{column}


\begin{column}{0.50\textwidth}

\begin{block}{Decomposition (space)}
\begin{equation}
\begin{aligned}
\boldsymbol{P}&=\textcolor{red}{L_P} (\boldsymbol{U}) \\
\boldsymbol{S}&= \textcolor{red}{L_S} (\boldsymbol{U})
\end{aligned}
\end{equation}
\end{block}

\begin{block}{Decomposition (wavenumber)}
\begin{equation}
\begin{aligned}
\boldsymbol{\tilde{P}}&=\mathrm{\bar{\boldsymbol{k}}} (\mathrm{\bar{\boldsymbol{k}}} \cdot \boldsymbol{\tilde{U}}) \\
\boldsymbol{\tilde{S}}&= \mathrm{\bar{\boldsymbol{k}}} \times (\mathrm{\bar{\boldsymbol{k}}} \times \boldsymbol{\tilde{U}})
\end{aligned}
\end{equation}
\end{block}
\end{column}
\end{columns}
\noindent{where a \textasciitilde represents data in the wavenumber domain, and a bar overhead means normalization.}
\end{frame}
%-----------------------------------------------------------
\begin{frame}{Methodology}
Only one of the spatial operators $\textcolor{red}{L_P}$ and $\textcolor{red}{L_S}$ needs to be calculated, as the other wavemode can be obtained by subtracting the (filteried) decomposed waves from the coupled waves component by component,
\begin{equation}
\boldsymbol{U}=\boldsymbol{P}+\boldsymbol{S}.
\end{equation}
\end{frame}
%-----------------------------------------------------------
\begin{frame}{Methodology}
The explicit expression for 2D isotropic P-wave decomposition in the wavenumber domain is (Zhang and McMechan, 2010)
\begin{equation}
\begin{aligned}
\tilde{U^P_x}&=\bar{k}^2_x \tilde{U_x} + \bar{k}_x\bar{k}_z\tilde{U_z}, \\
\tilde{U^P_z}&=\bar{k}^2_z \tilde{U_z} + \bar{k}_z\bar{k}_x\tilde{U_x}.
\end{aligned}
\end{equation}

To obtain the stationary filters for PS decomposition in the space domain, the wavenumber domain operators $\bar{k}^2_x$, $\bar{k}^2_z$ and $\bar{k}_x\bar{k}_z$ are transferred to the space domain by inverse FFTs to form spatial filters $L_x$, $L_z$ and $L_{xz}$, respectively.
\begin{equation}
\begin{aligned}
U^P_x&=L_x[U_x] + L_{xz}[U_z], \\
U^P_z&=L_z[U_z] + L_{xz}[U_x],
\end{aligned}
\end{equation}
and are plotted in the following slice...
\end{frame}
 
%%-----------------------------------------------------------
\begin{frame}{Filters}
   \begin{figure}[ht]
            \centering
            \includegraphics[width=\textwidth]{./Fig/filter.eps}
    \end{figure}
\end{frame}
%%-----------------------------------------------------------
\begin{frame}{Decomposition results (stationary filter)}
   \begin{figure}[ht]
            \centering
            \includegraphics[width=\textwidth]{./Fig/snap_filter.eps}
    \end{figure}
\end{frame}
%%-----------------------------------------------------------
\begin{frame}{Decomposition results (decoupled equation)}
   \begin{figure}[ht]
            \centering
            \includegraphics[width=\textwidth]{./Fig/snap_dp.eps}
    \end{figure}
\end{frame}
%%-----------------------------------------------------------
\begin{frame}{Decomposition results (wavenumber domain)}
   \begin{figure}[ht]
            \centering
            \includegraphics[width=\textwidth]{./Fig/snap_fft.eps}
    \end{figure}
\end{frame}
%%-----------------------------------------------------------
\begin{frame}{Computation cost}
\begin{table}[]
\centering
\label{my-label}
\begin{tabular}{|l|l|}
\hline
Methods                                                                    & Time (s/decomposition) \\ \hline
Stationary filter                                                          & 2.62                   \\ \hline
Decoupled equation                                                         & 0.046                  \\ \hline
\begin{tabular}[c]{@{}l@{}}Wavenumber domain\\  decomposition\end{tabular} & 0.22                   \\ \hline
\end{tabular}
\end{table}

\end{frame}
%%-----------------------------------------------------------
\section{Discussions}
\begin{frame}{Discussions}
\begin{itemize}
\item{The proposed spatial stationary filter gives acceptable PS decomposition results.}
\item{The computational cost for stationary filter is proportional to the size of filter. In this test we use a uniform size of 21 by 21.}
\item{The stationary filter is slower than decoupled equation and wavenumber domain decomposition methods for sequential computations.}
\item{The stationary filter gives more reliable decomposition results at model discontinuities (interfaces), as the latter generate serious artifacts.}
\item{The stationary filter is easier to implement for parallel computation than wavenumber domain decompositions, in which FFTs are performed.}
\end{itemize}
\end{frame}
%%%-----------------------------------------------------------
%\begin{frame}
%\begin{center}
%  \includegraphics[width=1.0\textwidth]{./Fig/Thankyou.eps}
%\end{center}
%\end{frame}

\end{document}

